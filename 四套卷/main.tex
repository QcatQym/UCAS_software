\documentclass{hm}
\workname{化反期中考试终极预测四套卷}
\name{中子哥}
\xuehao{2020E800418888}
\begin{document}
\qs{量纲分析}
搅拌槽的搅拌功率$P$与搅拌槽的直径$D$、液面高度$H$、搅拌桨直径$d_p$、搅拌转速$N$、液体密度$\rho$和粘度$\mu$有关:\[P=f(D,H,d_p,N,\mu,\rho)\]
\qs{系数矩阵}
确定下列反应中的独立反应
\begin{align*}
  2 \mathrm{C}_{2} \mathrm{H}_{2}+3 \mathrm{H}_{2} \mathrm{O}&\longrightarrow\mathrm{CH}_{3} \mathrm{COCH}_{3}+\mathrm{CO}_{2}+2 \mathrm{H}_{2} \\
  \mathrm{C}_{2} \mathrm{H}_{2}+\mathrm{H}_{2} \mathrm{O}&\longrightarrow\mathrm{CH}_{3} \mathrm{CHO} \\
  2 \mathrm{CH}_{3} \mathrm{CHO}+\mathrm{H}_{2} \mathrm{O}&\longrightarrow\mathrm{CH}_{3} \mathrm{COCH}_{3}+\mathrm{CO}_{2}+2 \mathrm{H}_{2} \\
  \mathrm{CH}_{3} \mathrm{CHO} &\longrightarrow \mathrm{CH}_{4}+\mathrm{CO} \\
  \mathrm{CO}+\mathrm{H}_{2} \mathrm{O}&\longrightarrow\mathrm{CO}_{2}+\mathrm{H}_{2} \\
  \mathrm{C}_{2} \mathrm{H}_{2}+4 \mathrm{H}_{2} \mathrm{O}&\longrightarrow2 \mathrm{CO}_{2}+5 \mathrm{H}_{2} \\
  \mathrm{CH}_{4}+3 \mathrm{CO}_{2} &\longrightarrow 4 \mathrm{CO}+2 \mathrm{H}_{2} \mathrm{O}
\end{align*}
\qs{复杂反应}
氨与甲醇生成一甲氨和二甲胺为连串反应:
\begin{align*}
  \mathrm{NH}_{3}+\mathrm{CH}_{3} \mathrm{OH} &\stackrel{k_{1}}{\longrightarrow} \mathrm{CH}_{3} \mathrm{NH}_{2}+\mathrm{H}_{2} \mathrm{O} \\
  \mathrm{CH}_{3} \mathrm{NH}_{2}+\mathrm{CH}_{3} \mathrm{OH} &\stackrel{k_{2}}{\longrightarrow}\left(\mathrm{CH}_{3}\right)_{2} \mathrm{NH}+\mathrm{H}_{2} \mathrm{O}
\end{align*}
均为不可逆反应,且为一级反应,某一温度下,氨转化率为0.7时一甲氨的收率最大,求该温度下$k_2/k_1$的值
\qs{简单反应}
可逆一级反应$A\overset{k_2}{\underset{k_1}{\rightleftharpoons}}B$在常温下进行,起始时只有反应物A,在T, 2T, 3T 时刻测定了A的浓度,其转化率分别为0.8, 0.7,0.65.求B的平衡浓度。

\textbf{回复爷获取答案}

\end{document}
